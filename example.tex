\documentclass[a4paper]{article}

\usepackage{amsfonts,amsmath,amssymb,graphicx,xcolor}
\usepackage{polyglossia}[numerals=mashriq, calendar=hijri]
\setmainlanguage{arabic}
\setotherlanguage{english}

\newfontfamily\arabicfont[Script=Arabic]{Amiri}

\NeedsTeXFormat{LaTeX2e}
\ProvidesPackage{arlatex}[2021/09/05 Arabic LaTeX package]

\RequirePackage{unicode-math}
\RequirePackage{graphicx}
\RequirePackage{fontspec}
\RequirePackage[utf8]{inputenc}


\title{\textenglish{Arabic in \LaTeX}}
\author{محمد نبيل}
\date{\textarabic{رمضان 1442}}

\begin{document}
	
\maketitle
	
\section{مقدمة}
	
	الرياضيات نشاط إنساني جوهري يمكن ممارسته وفهمه بطرق شتَّى. صحيحٌ أن الأفكار الرياضية نفسها أبعد ما تكون عن الجمود، ولكنها تتغيَّر وتتكيَّف مع مرورها عبر الفترات التاريخية والثقافات، وفي هذا الكتاب من سلسلة «مقدمة قصيرة جدًّا»، تَستكشِف جاكلين ستيدال التنوُّعَ التاريخي والثقافي الثري لمساعي علم الرياضيات، من الماضي البعيد وحتى يومنا هذا.
	
	من خلال عددٍ من دراسات الحالة المُشوِّقة المأخوذة من نطاقٍ من الأزمنة والأماكن، بما فيها الصين الإمبراطورية المُبكرة، والعالَم الإسلامي في العصور الوسطى، وبريطانيا في القرن التاسع عشر؛ يُلقِي هذا الكتابُ الضوءَ على بعض السياقات المختلفة التي تَعلَّمَ فيها الناسُ الرياضياتِ واستخدموها وورَّثوها.

	${\color{red}\arPi}$ هو نصف القطر
	
		
	\section{الحياء}
	كيف يمكن كتابة معادلة بالعربية
	$$\lambda=\sqrt{\theta}+\frac{F}{\alpha}$$
	إذا نظرنا إلى الدالة
	$\lambda$
	نجد أنها مشتقة من اليونانية 
	\textenglish{the greek alphabet}
	و هي لغة ميتة
	\begin{equation}
		\lambda=\sqrt{\theta}+\frac{F}{\alpha}
	\end{equation}

	\begin{equation}
		\lambda=\arSqrt{\Alef-\Beh}
	\end{equation}

	\begin{equation}
		\lambda=\sqrt{\theta} +\frac{\Alef}{\alpha}
	\end{equation}


	\begin{equation}
		\Dal=\Sad^{\Sheen}+\frac{\arPi}{\Seen}
	\end{equation}
	
	\begin{equation}
		\arSum{\Alef}{\Seen}{(\Jeem-\Reh)^{5\Noon}}
	\end{equation}
	
	
		هذه النظرية تنص على
		الإسلامي في العصور الوسطى، وبريطانيا في القرن التاسع عشر؛ يُلقِي هذا الكتابُ الضوءَ على بعض السياقات المختلفة التي تَعلَّمَ فيها الناسُ الرياضياتِ واستخدموها وورَّثوها.
	
\end{document}